\chapter{Shortest Path}

Given a weighted graph, one interesting problem is to find the
path with minimal weight.

\section{Single Source}

This instance of the shortest path problem begins at a specific vertex
in the graph, hence ``Single Source.''

\subsection{Dijkstra's Algorithm}

Given an acyclic graph $G=(V,E)$ in which every edge has a positive
weight and a single vertex $s$ from which to start the path, we begin
by labeling all vertices $u \in E | weight(u) = \infty$, then label
$weight(s) = 0$.  We also create a min-heap of the vertices using
weight as key.

We pop-best a vertex $v$ from our heap.  For each edge $e$ incident to
$v$ and some other vertex $u$, we check if $weight(v) + weight(e)$ is
less than $weight(u)$, in which case we set $weight(u) = weight(v) +
weight(e)$ and decrease the key of $u$ in our heap.  We continue this
process until no vertices remain in the heap.

It should be easy to see that if we have a particular target vertex,
we can halt the algorithm as soon as our target is the root of the
heap, because we have found the shortest path to it.

\subsection{Analysis of Dijkstra's Algorithm}

Intuitively, this algorithm has an upper bound of $O(|V|)$ times
however long it takes us to extract the minimum vertex from our heap,
plus $O(|E|)$ times however long it takes us to decrease the key,
since we have to check every vertex and we may have to check every
edge.  Using a standard heap, this gives us $O(|V|log|V| +
|E|log|V|)$.

\section{All Sources}

This instance of the shortest path problem is concerned with several
sources, in which case Dijkstra's Algorithm does not suffice.

\subsection{Floyd-Warshall Algorithm}

The Floyd-Warshall Algorithm is
\hyperlink{sec:floyd_warshall}{discussed in detail in the chapter on
  Dynamic Programming}.

\section{Arbitrary Weights}

A keen reader will have noticed that both Dijkstra's and Floyd
Warshall Algorithms assume positive edge weights.  Neither of which
are generally useful when there may be edges of negative weight.

\subsection{Bellman-Ford Algorithm}

The Bellman-Ford Algorithm can be applied with edges of arbitrary
weight.  Note that even Bellman-Ford doesn't deal with cycles of
negative weight, since these can be used to make any path have an
arbitrarily small total weight.

The Bellman-Ford Algorithm is beyond the scope of these notes, for
more information please see
\href{https://en.wikipedia.org/wiki/Bellman-Ford}{Wikipedia's entry on
  the Bellman-Ford Algorithm}, or CLRS.
