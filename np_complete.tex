\chapter{NP-Complete}

The complexity class known as \emph{NP-Complete} is a special subset
of $NP$, such that any problem in \emph{NP} can be translated to a
problem in \emph{NP-Complete} in polynomial time.

\section{CIRCUIT-SAT}

Given a boolean circuit, is it possible to provide a set of inputs
that cause the output to be \emph{True}?

In 1971, Cook proved that \emph{CIRCUIT-SAT} is \emph{NP-Complete}.
The proof is beyond the scope of these notes.  Cook showed that all
operations of a polynomial-sized Turing machine can be performed in
polynomial time using an instance of this problem.  In other words,
Cook showed that a Turing machine can be implemented using circuits
(surprise!!).

\section{Reduction}

Given at least one \emph{NP-Complete} problem, we can prove any other
problem $L$ is \emph{NP-Complete} by showing that $L \in NP$ and that
given an instance $x$ of a proven \emph{NP-Complete} problem we can
translate $x$ to an instance of $L$ in polynomial time, such that by
solving our generated instance of $L$, we can solve $x$.

The steps to do this are:

\begin{enumerate}
\item Show $L \in NP$.  We do this by first showing that the
  certificate for $L$ is polynomial with respect to the input, and
  that a certificate for $L$ can be verified in polynomial time.

\item Select a known \emph{NP-Complete} $L'$, ideally one that is
  similar to $L$.

\item Describe a polynomial time algorithm that maps any instance $x
  \in L'$ to an instance $f(x) \in L$.

\item Show that we can solve $x \in L'$ if and only if we can solve
  $f(x) \in L$.
\end{enumerate}

\section{SAT}

Given a boolean formula, are there values of the variables that cause
the formula to be \emph{True}?

First let us show that $SAT \in NP$ by using the truth values as a
certificate, which is clearly polynomial in size because it is
necessarily a subset of the input.  To verify, we evaluate the formula
given the truth values and return the output.

Next, we select \emph{CIRCUIT-SAT} from which to reduce to this
problem.

Then given an instance of \emph{CIRCUIT-SAT}, we transform the circuit
to a boolean formula such that the circuit is satisfiable if and only
if the formula is satisfiable.  We can do this easily by mapping gates
to boolean operators.

Since the instances of each problem are equivalent, it should be easy
to see that one is satisfiable if and only if the other is.

\section{3CNF-SAT}

Is a given boolean formula in \emph{3CNF} form satisfiable?  A boolean
formula is said to be in \emph{3CNF} form if there are no more than
$3$ variables in each clause, and within a clause there are only
\emph{OR} operators and between clauses there are only \emph{AND}
operators.

By the same proof as \emph{SAT} above, $3CNF-SAT \in NP$.

We reduce from \emph{SAT}.  Given an instance of \emph{SAT}, we use
equivalences to reduce all boolean operators to \emph{AND}, \emph{OR},
and \emph{NOT}.  Then we use DeMorgan's law to put the formula into
\emph{3CNF} form.

Since the instances of each problem are equivalent, it should be easy
to see that one is satisfiable if and only if the other is.

\section{CLIQUE}

\section{VERTEX-COVER}

\section{TSP}

\section{SUBSET-SUM}

\section{0/1-Integer Programming}

\section{3-COLOR}

