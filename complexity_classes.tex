\chapter{Complexity Classes}

A complexity class is a set of related problems.  Before we talk about
specific classes, it is important to understand the difference between
a decision problem and an optimization problem.

\section{Optimization Problems}

An optimization problem is a problem of the form ``find the optimal
solution for problem p.''  Each such problem may have a different
defintion of optimal, and there may be many optimal solutions.

\section{Decision Problems}

A decision problem is much more limited in scope.  These are problems
with strictly yes/no answers, often of the form ``is there a solution of
size (at least or at most) k to problem p?''.

Notice that optimization problems have related decision problems.  For
any optimization problem we can ask the decision problem ``is x the
optimal solution to problem p?''.

\section{P}

The complexity class $P$ is the set of all decision problems which are
solvable in polynomial time.  That is to say, all problems for which
there exists an algorithm to solve the problem bounded above by
$O(n^d)$ where $d \in \mathbb{Z}$.

\section{NP}

The complexity class $NP$ is the set of all decision problems for
which a certificate (possible solution) may be verified in polynomial
time.  

\section{$P \subseteq NP$}

It should be obvious that any problem in $P$ is also in $NP$, since if
we can find the answer in polynomial time without the certificate, we
must certainly be able to do so with the certificate.

\section{P = NP?}

It is not obvious if there are problems in $NP$ but not in $P$ or if
all problems in $NP$ are also in $P$.  In fact,
\href{http://www.claymath.org/}{the Clay Mathematics Institute} has
\href{http://www.claymath.org/millennium/P_vs_NP/}{listed this as one
  of the Millenium Prizes} for which the award for solving is 1
million USD.
