\chapter{Union-Find}

Consider the following problem. Given $n$ disjoint sets of 1 element each,
perform $n$ unions and then $m$ queries for what set a given element is in.
We will call these two operations $union(A,B)$ and $find(x)$. 

\section{Approach 1: Linked Lists}
Our first
approach to this problem is to describe our sets as linked lists. We know
we can combine linked lists quite quickly, so this seems ideal for $union$. 
All we must do is have the head of $B$ point to the tail of $A$, which can be
done in constant time.
However, linked lists are not particularly well suited for $find$. To resolve
this, at every node we shall store a $back pointer$ to the linked list 
the node is part of. This allows us to perform $find$ in constant time. 
However, now $union$ needs some extra work.
When we call $union(A,B)$, we will now walk through $B$ and fix all of its
back pointers to point to $A$. This will take time linear in the size of $B$.

Given $n$ sets, what is the worst possible way to $union$ them? Well, since
the run time $union$ is linear in the size of the second list, adversarial
we can
do is to $union$ every individual element with the current unioned set. For
instance, $union(D,union(C,union(A,B)))...$. Clearly this will require
$O(\summ{i=1}{n-1}i)$, which is $O(n^2)$. If we then perform $m$ $find$s, all
of which take $O(1)$ time, we will have performed $n$ $union$s and $m$ $find$s
in $O(n^2 + m)$ time.

\section{Approach 2: Better Linked Lists}
Somehow our first approach was naive, which allowed us to ``game" the system
to create a very bad result for the $union$s. To get a better result, we will
make a slight modification to our $union$ algorithm. Instead of blindly
attaching $B$ to the end of $A$, we will attach the smaller set to the larger.
This fixes our adversarial approach, but is it actually better? 

Consider how frequently we need to change the back pointers on an individual
node. At first, it is part of a set of size one, and will have to change its
pointer when unioned to a set of size $\geq 1$, placing it in a set
of size $\geq 2$. The next time it will be changed is when it is unioned
to a set of size $\geq 2$, then $\geq 4$ and so on. The last time will be
when it is unioned to a set of size $\geq n/2$ after which we can not find
another set of large enough size to change it again. Therefore each
back pointer needs to be changed $O(\log n)$ times at worst. Since there
are $n$ back pointers, this new approach only require $O(n \log n)$
time to perform the unions. Since the $find$ operation is not effected,
our approach now performs $n$ $union$s and $m$ $find$s
in $O(n \log n + m)$ time.

\section{Approach 3: Trees}
Having to fix back pointers is still fairly wasteful, what if we didn't have
to? Instead of implementing our sets as linked lists, we can instead
use trees. Each set will be a node with either the name of the set, or a
pointer to its parent. When we perform $union(A,B)$, we will simply
replace the name of the shorter set with a pointer to the head of the taller
set. Since we are just changing a pointer, this will only take $O(1)$ time.
Our $find$ algorithm, however, will now have to walk from the node all the way
to the root of the tree it is in to find out what list it is in. By a
similar argument from the previous section, the height of our tree will only
ever be $\log n$. 
Therefore this approach can support $n$ $union$s and $m$ $find$s
in $O(n + m \log n)$ time.

\section{Approach 4: Path Compression}
Our $union$ algorithm is optimal using the tree approach, but it has
made our $find$ algorithm suffer. To fix this, we make a simple observation.
Since our find algorithm must already walk through several nodes in the tree,
once we get to the root we can, without worsening the time,
relabel all of their pointers to point directly to the root. This approach
is called $path compression$. This will make
subsequent queries on these elements and their children substantially faster.
The analysis of this algorithm is beyond the scope of this class, but
evidently it  can support $n$ $union$s and $m$ $find$s
in $O(n + m \alpha(n))$ time. Where $\alpha$ is the inverse Ackermann function. 
Although $\alpha(n)$ tends towards infinity as $n$ does, for any ``practical"
$n$ it is at most $4$. It turns out that this is in fact optimal for the
union-find problem.
