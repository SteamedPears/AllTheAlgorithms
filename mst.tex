\chapter{Minimum Spanning Tree}

Given an undirected, connected graph where each edge has positive
weight, the \emph{Minimum Spanning Tree} (MST) is a connected subgraph on the same 
vertex set with minimal weight. Intuitively, the MST is the lightest possible
connected subgraph. The MST is not necessarily unique, 
for proof of this, consider a graph where all edges are of equal weight. 
Any tree is an MST of such a graph. 

\section{Kruskal's Algorithm}

%The idea of this algorithm is to begin with the vertex list and an
%empty edge list, and maintain a forest, adding the edge of minimum
%weight that does not make a cycle until the forests are connected.

Kruskal's algorithm for computing the MST relies on the simple observation
that the smallest edge in a graph $G$ is part of \emph{some} MST of $G$.
Kruskal's algorithm starts by constructing a min-heap containing all $m$ edges
in $G$, and a collection of disjoint sets, each containing one of the
$n$ vertices of $G$. We also construct an empty list $T$ that will hold
all the edges of the MST. We then get the minimum edge from the heap, check if
the two nodes the edge joins are from the same set, and if not add 
the edge to $T$, and $union$ the two sets that the edge joins.
We repeat this until $T$ contains $n-1$ edges. 

Because this algorithm uses
structures we already know and understand, analysis will be fairly easy.
We require $O(m) + O(n)$ time to construct the initial sets and heap. We also
require $O(m \log m)$ to extract the minimums from the heap. Our $n-1$ unions
and $n-1$ finds can be done in $(n \log n + n)$ time. Therefore this algorithm
takes $O(m \log m + n \log n)$.

\section{Prim's Algorithm}
Prim's algorithm for computing the MST is fairly similar to Kruskal's. However,
instead of working with all of the edges and vertices at once, it picks one
vertice and builds from that. We start by constructing an empty list $A$ which
will hold all the vertices that are part of the MST so far, a list $T$ which
will hold all the edges in the MST so far, a min-heap $V$ which will contain all
the vertices not in $A$. At first every node is given a key of infinity. 
We then pick a random vertex $v$ from $V$ and move it from $V$ 
to $A$. Next we look at all the edges of $v$ and set the
keys of the corresponding nodes to the weight of these edges. Now we retrieve
the minimum node $u$ from $V$ and move it to $A$, and its key edge to $T$. 
Next we look at all the edges on $u$ and if their weight is smaller than
the current key of their corresponding edge, set the key to that weight. Then
we simply retrieve another node from the heap and repeat. After $n-1$ iterations
we will have a complete MST.

This algorithm requires $O(n)$ time to construct the initial heap and 
initialize the first node, and since we must update at most $m$ keys in the
heap of $n$ elements, it requires $O(m \log n)$ time to do this. Therefore the
entire algorithm takes $O(n + m \log n)$ time.

